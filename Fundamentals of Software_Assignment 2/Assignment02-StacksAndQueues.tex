\documentclass[12pt]{elsart}
\usepackage{amsmath}
\usepackage{amssymb}
\usepackage{program}
\newcommand{\field}[1]{\mathbb{#1}}

\usepackage{algorithm}
\usepackage{algpseudocode}

\usepackage{enumitem}

%%%%%%%%%%%%%%%%%%%%%%%%%%%%%%%%%%%%%%%%%Space to make more readable!
%\vspace{10 mm}
%%%%%%%%%%%%%%%%%%%%%%%%%%%%%%%%%%%%%%%%%Take out later!

\begin{document}

\pagestyle{empty}

\begin{center}
\Large  CS3343 Analysis of Algorithms Fall 2020 \\
\large {\bf Homework 2}\\
\normalsize Due 9/30/20 before 11:59pm (Central Time)
\end{center}

{\bf 1. Evaluating Postfix Notation (20 points)}

Sketch of algorithm for evaluating postfix strings:

\begin{enumerate}
   \item Create stack $s$.
   \item For each token, $x$, in the postfix expression:
\begin{enumerate}[label=\arabic*]
   \item If $x$ is T or F push it into the stack $s$.
   \item Else if $x$ is a unary operator
\begin{enumerate}[label=\roman*]
   \item pop an operand, $op1$, from $s$
   \item compute $x$ $op1$ (see unary table below)
   \item push the result into $s$
\end{enumerate}
   \item Else if $x$ is a binary operator
\begin{enumerate}[label=\roman*]
   \item pop an operand, $op2$, from $s$
   \item pop an operand, $op1$, from $s$
   \item compute $op1$ $op2$ $x$ 
   \item push the result into $s$
\end{enumerate}
\end{enumerate}
   \item If you do not have enough operands in $s$ to perform an operation you should return an error in the boolean.
   \item Likewise, if $s$ contains more than one operand after all of the tokens are evaluated you should return an error in the boolean.
   \item Otherwise pop and return the only value in $s$.
\end{enumerate}


\begin{center}
  \begin{tabular}{ | p{3.5cm} | p{4cm} | p{4cm} | }
    \hline
    {\bf Operator Type} & {\bf Usage} & {\bf Calculation}  \\ \hline \hline
    unary operator & $op1$ NOT &  $!op1$ \\ \hline
    binary operator & $op1$ $op2$ AND &  $op1$ \&\& $op2$ \\ 
    binary operator & $op1$ $op2$ NAND & $!(op1$ \&\& $op2)$ \\ 
    binary operator & $op1$ $op2$ OR & $op1$ $\vert \vert$ $op2$ \\ 
    binary operator & $op1$ $op2$ NOR & $!(op1$ $\vert \vert$ $op2)$ \\ 
    binary operator & $op1$ $op2$ XOR & $op1$ $!=$ $op2$ \\ 
    binary operator & $op1$ $op2$ COND & $!op1$ $\vert \vert$ $op2$ \\ 
    binary operator & $op1$ $op2$ BICOND & $op1$ $==$ $op2$ \\ \hline
  \end{tabular}
\end{center}

\newpage

{\bf 2.  Probability (7 points)}

For the following problem, clearly describe the sample space and the random variables you use.  Be sure to justify where you get your expected values from.

Consider playing a game where you roll $n$ fair six-sided dice.  For every 1 or 6 you roll you win \$30, for rolling any other number you lose \$$3 n$ (where $n$ is the total number of dice rolled).

\begin{enumerate}
   \item  First assume $n=1$ (i.e., you only roll one six-sided die). 
\begin{enumerate}
   	\item  (1 point) Describe the sample space for this experiment.
  	 \item (1 point) Describe a random variable which maps an outcome of this experiment to the winnings you receive.
  	 \item (1 point)  Compute the expected value of this random variable.
\end{enumerate}
   \item  Now assume $n=7$ (i.e., you roll 7 six-sided dice). 
\begin{enumerate}
   	\item (1 point) Describe the sample space for this experiment (you don't need to list the elements but describe what is contained in it).
   	\item (1 point) Describe a random variable which maps an outcome of this experiment to the winnings you receive. ({\bf Hint:} Express your random variable as the sum of seven random variables.)
 	  \item  (1 point) Compute the expected value of this random variable using the linearity of expectation.  Based on this would you play this game?
\end{enumerate}
   \item  (1 point) What is the largest value of $n$ for which you would still want to play the game?  Justify your answer.
\end{enumerate}

{\bf 3. Expected Runtime - Linear search (6 points)}

The code below searches an unsorted array, $A$, for a given value.  It returns the index of the value if it is present.  Otherwise it returns $-1$.

\begin{algorithm}
\caption{int linearSearch(int $A[0\ldots n-1]$, int $val$)}
 \begin{algorithmic}
 \State $i = 0$;
 \While{$i< n$}
    \If{$val==A[i]$}
       \State return $i$;
    \EndIf
    \State $i++$;
  \EndWhile
 \State return $-1$;
\end{algorithmic}
\end{algorithm}

Suppose before we call linearSearch we randomly permute (i.e., reorder) the elements in $A$.  In the following problem, we will find the expected asymptotic runtime of linearSearch.

\begin{enumerate}
 \item (1 point) What is the best case asymptotic runtime of linearSearch on an array of $n$ elements?  What index must $val$ be located at to elicit this runtime?
 \item (1 point) What is the worst case asymptotic runtime of linearSearch on an array of $n$ elements?  What index must $val$ be located at to elicit this runtime?
 \item (1 point) For the sake of simplicity, assume $val$ is in $A$.  Let $X_j$ be the random variable which, given an permutation of the elements in our array, is $1$ if index $A[j]$ will be checked in our linearSearch for that permutation and $0$ otherwise.  What is the expected value of $X_j$?
 \item (1 point) Let $X$ be the random variable which, given an permutation of the elements in our array,  is the number of elements of $A$ which will be checked in our linearSearch for that permutation.  Define $X$ using $n$ copies of the random variable defined in the previous step.
 \item (2 points) Compute the expected value of $X$ for using the linearity of expectation. Given this, what is the expected asymptotic runtime of linearSearch?
\end{enumerate}

{\bf 4. Expected Runtime - OTPP (6 points)}

Let's suppose you've been hired to help design a new game show called ``One True Programming Pair''.  The participants will compete to see which pair of programmers are the best at cooperative coding.

Your job is to take an array of programmers and find a set of candidate pairs that will compete in the game show.  Inspired by the hireAssistant algorithm you develop the following algorithm.  It adds a pair to the candidate list if the pair is better than the best pair seen so far.  

\begin{algorithm}
\caption{void findCandidatePairs( person $A[1\ldots n]$ )}
 \begin{algorithmic}[1]
 \State $best = -1$;   //This will contain core of the best pair seen so far
 \State $i = 1$;
 \While{$i\leq n-1$}
    \State $j = i+1$;
    \While{$j \leq n$}
       \State //Check if pair i,j is better than best pair seen so far
       \If{$best < evalPair(A[i],A[j])$}
          \State $best = evalPair(A[i],A[j])$;
          \State Add the pair $i,j$ to to list of candidate pairs 
          \State //Note: a person is permitted to be in multiple candidate pairs
       \EndIf
       \State $j++$;
    \EndWhile
    \State $i++$;
  \EndWhile
\end{algorithmic}
\end{algorithm}

\begin{enumerate}
 \item (1 point) What is the smallest number of times that line 9 will run for findCandidatePairs on an array $A$ of $n$ people?  
 \item (1 point) What is the largest number of times that line 9 will run for findCandidatePairs on an array $A$ of $n$ people?   
 \item (1 point) Define the random variable $X_{ij}$ as follows:
\[
 X_{ij}(A)= \begin{cases} 
      1 & \text{For array $A$ the pair $i$ and $j$ are added to the candidate list} \\
      0 & \text{Otherwise}
   \end{cases}
   \]
 Let's assume the probability that $X_{ij}=1$ is $\frac{1}{ij}$. Compute $E(X_{ij})$.
 \item (1 point) Let $X$ be the random variable which, given our array $A$, which is the number of pairs added to our list of candidates.  Define $X$ using a nested sum of the random variable $X_{ij}$.
 \item (2 points) Compute $E(X)$.\\
{\bf Hint:} $\sum\limits_{x=s_1}^{f_1}\sum\limits_{y=s_2}^{f_2}A_xB_y=\left(\sum\limits_{x=s_1}^{f_1} A_x\right)*\left(\sum\limits_{y=s_2}^{f_2}B_y\right)$ where $s_1\leq f_1$, $s_2 \leq f_2$, and they are all integers.  $A_x$ is a function on $x$ and is not related to $y$.  Likewise $B_y$ is a function on $y$ and is not related to $x$.  
\end{enumerate}

\end{document}